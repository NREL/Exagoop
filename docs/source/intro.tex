\section{Introduction}
\label{sec:intro}
Disposal of concentrated brines is a significant contributor to operating costs in industrial desalination and current methods to achieving zero-liquid/minimum-liquid discharge (ZLD/MLD) is through energy and carbon intense thermal desalination. An alternative energy and cost efficient pathway is through Ultra-High Pressure Reverse Osmosis (UHPRO) that can operate at high brine concentrations on the order of 200-250 g/L. The high osmotic pressure requirement for UHPRO leads to operating pressures on the order of 200 bar compared to traditional salt water RO that requires $\sim$ 70 bar. RO operation at very high pressure ($\sim$ 200 bar) is challenging due to the degradation of membrane performance from 1) mechanical compaction of Polysulfone (PSF) and polyester support layer thus reducing porosity and permeability, 2) embossing of the membrane into permeate spacers that impedes permeate flow and increases pressure losses and 3) increased concentration polarization in viscous hyper-saline brines that leads to reduced mass-transfer. 

In order to address these challenges, an improved understanding of the mechanisms of compaction and embossing in spiral wound elements at UHPRO operating pressures is required. This report documents the development of a computational model for membrane structural mechanics using the material-point-method (MPM), using which membrane deformation under high pressure is simulated. Our simulations agrees well with experimental measurements on the overall displacement of the PSF layer and qualitatively shows the reduction in porosity through macrovoid closure. This report is organized as follows. Section \ref{} describes the model equations and the MPM algorithm used to solve them. Section \ref{} describes the verification of our solver with canonical solid mechanics problems, Section \ref{} presents the validation of our simulation methodology with experimental measurements of membrane compaction under pressure. 
% Technical barriers and targets
% State-of-the-art SWRO membranes are limited to 70 bar and emerging HPRO membranes are limited to 80-120 bar (depending on temperature) due to compaction and embossing.
% UHPRO membranes stable up to 200 bar are needed to achieve MLD with lower energy and cost than MVC and other thermal brine concentration processes
% Commercially viable materials of construction to achieve up to 200 bar tolerant UHPRO membranes will be developed
% Algorithms and software that enable reliable simulation and optimization of UHPRO membranes, modules and systems will be developed and made available for general use
%51% of industrial desalination OPEX is brine disposal
%Ultra-high pressure RO membranes and modules tolerating up to 200 bars could reduce the energy and cost of brine concentration by as much as 50% relative to MVC and other thermal brine concentration technologies.
%However, state-of-the-art RO membranes suffer dramatic performance decline and damage due to (1) physical membrane compaction, (2) embossing into permeate tricot mesh spacers and (3) enhanced CP due to poor mass transfer in viscous hyper-saline brines.
% Due to the high osmotic pressure of brines, thermal brine concentration remains the only way to achieve minimum, and ultimately, zero liquid discharge (MLD/ZLD), but it remains very energy intense (12-20 kWh/m3) and expensive ($5/m3).
% State-of-the-art SWRO membranes are limited to 70 bar and emerging HPRO membranes are limited to 80-120 bar (depending on temperature) due to compaction and embossing.
% UHPRO membranes stable up to 200 bar are needed to achieve MLD with lower energy and cost than MVC and other thermal brine concentration processes
% Establish a comprehensive understanding of the mechanisms of compaction and embossing in spiral wound elements at operating pressures up to 200 bar
% Develop improved models to explain, predict and optimize water and salt transport through RO membranes and spiral-wound modules at salinities up to 250 g/L and operating pressures up to 200 bar
% Establish new, commercially-manufacturable membrane chemistries and structures in addition to new feed/permeate spacer structures to optimize UHPRO performance at pressures up to 200 bar



%Intro to numerical simulations
%\paragraph{}
Numerical simulations have become an indespensible part of engineering analysis today. They are a good substitute to costly experiments and are often used to downselect prime designs for performing experimental tests. They are also desirable due to the availability of numerical datasets that can shed great insights to spatial and temporal scales which are hitherto inaccessible in experiments. Numerical simulations of HPRO membrane compaction and flow in HPRO devices have been attempted in the past \cite{Gu2017,Pankaj2016,Abdelbaky2019,Liang2019,Lelong2019,Mao2021,Benjamin2022,Aschmoneit2022}. Recently, numerical simulations are even used in computer animations to be used in movies such as \textit{Frozen, Big Hero 6} and \textit{Zootopia}.
%Literatue review
 
%What are the different numerical methods in use today?
The conventional and the oldest numerical methods applied to continuum mechanics are based on either Finite Volume Methods(FVMs), Finite Element Methods (FEMs) or Finite Difference Methods (FDMs). While the first two methods are based on the integral form of the governing quations, the last one is based on the differential form by approximating the derivatives using algebraic expressions. An aspect which is common to the above three methods is the need for a computational grid to solve the governing equations. The grid refers to a collection of numbered points (or nodes) which are connected using edges (in 2D) or faces (in 3D) to form the full computational domain. It is common in solid mechanics to use the above class of methods in conjunction with Lagrangian form of the governing equations. In such applications, the grid nodes are attached to the material and move as the material deforms. When the material undergoes severe deformation, the mesh nodes move along with the material and lead to grid entanglement. This ultimately leads to numerical instabilities and the solver stopping abrptly. On the other hand, Eulerian description based methods do not require grid movement as the material advects across grid cells and are not attached to grid nodes. Hene, this formulation although numerically stable, is not a good candidate for problems with moving interfaces and with materials with history dependant properties. 

%MPM vs other methods

Material point method (MPM) \cite{SULSKY1994179, Bardenhagen2004, Zhang2015TheMP, Vaucorbeil} is a 'mesh-free' method that has received a lot of attention recently for its ability to simulate problems with severe deformations. MPM is a Lagrangian, particle-based numerical method developed by drawing inspiration from the particle-in-cell (PIC) \cite{osti_4769185} and fluid implicit particle method (FLIP) \cite{BRACKBILL1986314} methods. Although MPMs were initially applied to study solid mechanics problems, it has been considerably extended to be applied to fluid simulations as well. In MPM method, the full material domain is descretised using particles or material points. All the material properties such as the velocity, density, strainrates and stresses are stored at the material points. This makes this method attractive to model history dependant constitutive models. The lack of a grid connecting these material points make MPM suitable to simulate problems with severe material deformations such as in solid fracture, foam deformations and in granular flows. Although a background grid is necessary in MPM, it is simply used as a scratch pad to carry out certain numerical operations that are not computationally intensive. These characteristics of MPM make it a good candidate for studying membrane compaction problems.